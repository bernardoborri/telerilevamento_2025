Qui per i commenti, essendo in linguaggio LaTeX e non in linguaggio R, usiamo il simbolo % anziché #
Qui le funzioni iniziano col backslash e gli argomenti sono tra parentesi graffe
\documentclass[12pt]{article} % con le parentesi quadre cambio la struttura, in questo caso la dimensione dei caratteri
\usepackage{graphicx} % pacchetto necessario per inserire immagini nel mio documento su Overleaf
\usepackage{hyperref} % pacchetto necessario per inserire dei collegamenti tra diversi punti del documento
\usepackage{float} % pacchetto necessario a mettere le immagini esattamente dove le voglio io e non dove vuole LaTeX
\usepackage{natbib} % pacchetto necessario ad inserire una bibliografia alla fine del mio documento
\usepackage{lineno} % pacchetto necessario  ad aggiungere dei numeri di riga (necessari per esempio ad un revisore di un articolo per indicare dove c'è un errore)
\linenumbers

% esempio: questo è un commento
\title{Il mio primo documento con LaTeX}
\author{borri.pachypus}
% \date{}

\begin{document} % ovunque si usi la funzione "begin" serve chiudere alla fine con la funzione "end" altrimenti da errore!

\maketitle

\tableofcontents

\section{Introduction} \label{sec:intro}
% uso la funzione "label" per poter richiamare l'introduzione nei paragrafi successivi
% inserisco il testo di una canzone per mettere una roba esemplificativa nell'introduzione
% scelgo il brano della grande artista Cicciolina dal titolo "Muscolo rosso"
[Testo di "Muscolo rosso"]

[Strofa 1]
Dopo le mie trasgressioni
Dopo tutte queste emozioni
Nessuno mi può fermare
Non mi potete arrestare

[Strofa 2]
Selvaggio animale in calore
Il cazzo che mi spruzza nel cuore
Un muscolo rosso d'amore
Affonda lungo il mio cuore

[Pre-Ritornello]
Tu, che sembri un manichino
Tira fuori il cazzo duro
Ti faccio un pompino
Io ti faccio un pompino, oh

[Ritornello]
Voglio il cazzo
Vestita di pelle
Il cazzo
Più duro del muro
Il cazzo
Nel buco del culo
Il cazzo che mi sfonderà, ah
Insieme a me schizzerà
Voglio il cazzo
Vestita di pelle
Il cazzo
Più duro del muro
Il cazzo
Nel buco del culo
Il cazzo che mi sfonderà, ah
Insieme a me schizzerà
In mio potere sarà

\section{Methods} 
% qui, per esempio, ci metto un'equazione a caso (gravità) 
% per illustrarla cerco sul web le funzioni necessarie ad inserire le operazioni matematiche, facilmente reperibili
In this paper we relied on Equation \ref{eq:newton}:

\begin{equation}
F = G \times \frac{m_1 \times m_2}{d^2}
\label{eq:newton}
\end{equation}

The equation has then been changed to Equation 2:

\begin{equation}
F = \sqrt{\sum_{i=1}^{N} G \times \frac{\sqrt{m_1 \times \sqrt{m_2}}}{d^{\sqrt[3]{2}}}}
\end{equation}

All of this according to what we said in section \ref{sec:intro}.
% ho qua sopra richiamato con la funzione "label" l'intoduzione

\section{Results}

% con l'insieme di funzioni qua sotto inserisco un'immagine, la centro, ne determino la misura e gli faccio comparire una descrizione sotto
% gli ho anche dato un'etichetta con la già vista funzione "label"

\begin{figure}[H]
    \centering
    \includegraphics[width=\linewidth]{ggplot.png}
    \caption{Output graph of the analysis showing a temporal trend.}
    \label{fig:temporal}
\end{figure}

\begin{thebibliography}{999}

\bibitem[Cramieri et al. (2020)]{cramieri}
Crameri, F., Shephard, G. E., \& Heron, P. J. (2020). The misuse of colour in science communication. Nature communications, 11, 5444.

\bibitem[Chabreck \& Joanen (1979)]{alligator}
Chabreck, R. H., \& Joanen, T. (1979). Growth rates of American alligators in Louisiana. Herpetologica, 51-57.

\end{thebibliography}

\end{document}
