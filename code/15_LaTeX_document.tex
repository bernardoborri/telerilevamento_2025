Qui per i commenti, essendo in linguaggio LaTeX e non in linguaggio R, usiamo il simbolo % anziché #
Qui le funzioni iniziano col backslash e gli argomenti sono tra parentesi graffe
\documentclass[12pt]{article} % con le parentesi quadre cambio la struttura, in questo caso la dimensione dei caratteri
\usepackage{graphicx} % pacchetto necessario per inserire immagini nel mio documento su Overleaf
\usepackage{hyperref}
\usepackage{natbib}
\usepackage{float}

\usepackage{lineno}
\linenumbers

% esempio: questo è un commento
\title{Il mio primo documento con LaTeX}
\author{borri.pachypus}
% \date{}

\begin{document} % ovunque si usi la funzione "begin" serve chiudere alla fine con la funzione "end" altrimenti da errore!

\maketitle

\tableofcontents

\section{Introduction} \label{sec:intro}
[Testo di "Muscolo rosso"]

[Strofa 1]
Dopo le mie trasgressioni
Dopo tutte queste emozioni
Nessuno mi può fermare
Non mi potete arrestare

[Strofa 2]
Selvaggio animale in calore
Il cazzo che mi spruzza nel cuore
Un muscolo rosso d'amore
Affonda lungo il mio cuore

[Pre-Ritornello]
Tu, che sembri un manichino
Tira fuori il cazzo duro
Ti faccio un pompino
Io ti faccio un pompino, oh

[Ritornello]
Voglio il cazzo
Vestita di pelle
Il cazzo
Più duro del muro
Il cazzo
Nel buco del culo
Il cazzo che mi sfonderà, ah
Insieme a me schizzerà
Voglio il cazzo
Vestita di pelle
Il cazzo
Più duro del muro
Il cazzo
Nel buco del culo
Il cazzo che mi sfonderà, ah
Insieme a me schizzerà
In mio potere sarà

\section{Methods} 
% qui, per esempio, ci metto un'equazione a caso (gravità) e per illustrarla 
cerco sul web le funzioni necessarie ad inserire le operazioni matematiche, facilmente reperibili
In this paper we relied on Equation \ref{eq:newton}:

\begin{equation}
F = G \times \frac{m_1 \times m_2}{d^2}
\label{eq:newton}
\end{equation}

The equation has then been changed to Equation 2:

\begin{equation}
F = \sqrt{\sum_{i=1}^{N} G \times \frac{\sqrt{m_1 \times \sqrt{m_2}}}{d^{\sqrt[3]{2}}}}
\end{equation}
